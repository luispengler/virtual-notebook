\documentclass{article}
\title{Controle de motor de esteira com sensor capacitivo}
\date{}

\usepackage[utf8]{inputenc}
\usepackage[portuguese]{babel}
\usepackage[margin=2.5cm]{geometry}
\usepackage{amsmath}
\usepackage{physics}
\usepackage{pgfplots}
\pgfplotsset{compat = newest}
\usepackage{titlesec}
\usepackage{graphicx}
\usepackage{wrapfig}
\usepackage{caption}
\usepackage{subcaption}
\usepackage{tabularx}
\usepackage[parfill]{parskip}
\usepackage[nottoc]{tocbibind}
\usepackage[backend=biber]{biblatex}
\addbibresource{/home/luispengler/drive/LinuxFabrik/Research/read/bib.bib}
\usepackage{authblk}
\renewcommand\Authand{ e }
\renewcommand\Authands{ e }
\author[1]{Luís Guilherme Miranda Spengler}
\author[2]{Diogo Paes Masacottes}
\affil[1,2]{Instituto Federal de Educação, Ciência e Tecnologia de Mato Grosso do Sul}

\begin{document}
\maketitle

\section{Introdução}
Seguindo o relatório intitulado ``Controle de máquina motora monofásica'', substituimos o acionamento por chave fim de curso por um sensor capacitivo, a fim de ter um atuador operando conforme uma certa condição se faça verdadeira. É grande a utilidade de sistemas de controle que operam conforme uma certa condição se faça verdadeira e até se tornam indispensáveis, já que qualquer sistema de controle simples funciona com base neste princípio. Neste presente artigo é apresentado uma situação em que o controle de uma esteira pode ser efetuado por um sensor capacitivo.

\section{Problemática}

Adentramos no problema considerando uma esteira que realiza o transporte de lixo em uma operação para a reciclagem de lixo encontrado em uma praia. Com o funcionamento do motor que move a esteira, há o aquecimento de seus condutores, principalmente os do enrolamento da bobina, ocasionando em redução da vida útil se o aquecimento for demasiado. Também há consumo de energia elétrica ao se utilizar um motor, principalmente maior potência reativa, gerando gastos extras aos donos da instalação.

\section{Objetivo Geral}

O objetivo é realizar o controle conforme detecção de lixo, para que se positivo a detecção de lixo, o motor acione e a esteira se movimente.

\section{Metodologia}

A figura 1 representa o circuito performado, na prática, onde Q1 corresponde a um disjuntor, KM1 corresponde a ligação feita na contatora. O primeiro componente no lado Vcc é a chave que liga o inversor, logo em seguida o sensor capacitivo. Por último, do lado Vcc, estão as bobinas da contatora. M1 corresponde ao motor 1 da esteira, que só estará ligado se a chave do circuito estiver fechada e o sensor capacitivo fechar o circuito, permitindo a contatora ficar com a chave fechada.

\begin{figure}[h!]
\centering
\def\svgwidth{1\columnwidth}
\input{ligacao.pdf_tex}
\caption{Esquema do circuito}
\end{figure}

\section{Resultados}

Porque o motor liga se o sensor capacitivo detectar algo qualquer (lixo) e se não detectar e estiver funcionando, o motor desliga, pode-se deduzir o tipo de controle como de ação liga-desliga (on-off).

\section{Conclusão}

Foi possível identificar, esquematizar e montar um circuito de controle que fizesse com que uma esteira operasse ou não dependendo do estado de um sensor capacitivo.

\end{document}
