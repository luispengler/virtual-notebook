\documentclass{article}
\title{Sistemas Digitais e Microcontrolados - Relatório de Laboratório em Display de 7 Segmentos}
\date{}

\usepackage[utf8]{inputenc}
\usepackage[portuguese]{babel}
\usepackage[margin=3.5cm]{geometry}
\usepackage{amsmath}
\usepackage{physics}
\usepackage{titlesec}
\usepackage{graphicx}
\usepackage{wrapfig}
\usepackage{caption}
\usepackage{subcaption}
\usepackage[parfill]{parskip}
\usepackage[nottoc]{tocbibind}
\usepackage[backend=biber]{biblatex}
\addbibresource{/home/luispengler/drive/LinuxFabrik/Research/read/bib.bib}
\usepackage{authblk}
\author[1]{Ana Julia Alencar}
\author[2]{Luís Spengler}
\author[3]{Lukas Oshima}
\affil[1,2,3]{Instituto Federal de Educação, Ciência e Tecnologia de Mato Grosso do Sul}

\graphicspath{{./docs/}}

\begin{document}
\maketitle

\tableofcontents

\medskip

\section{Introdução}
Fomos introduzidos ao display de 7 segmentos, este sendo o modelo WCN1-10565R-C13RA


\section{Conclusão}

Por fim, pode-se concluir que todos os gates (portas) testados experimentalmente neste relatório estavam funcionando corretamente. Até mesmo sua implementação na criação de uma porta NAND funcionou conforme esperado.

\medskip

\end{document}
