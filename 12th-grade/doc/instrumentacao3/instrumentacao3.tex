\documentclass{article}
\title{Identificação de histerese em um Esfigmomanômetro}
\date{}

\usepackage[utf8]{inputenc}
\usepackage[portuguese]{babel}
\usepackage[margin=3.5cm]{geometry}
\usepackage{amsmath}
\usepackage{physics}
\usepackage{pgfplots}
\pgfplotsset{compat = newest}
\usepackage{titlesec}
\usepackage{graphicx}
\usepackage{wrapfig}
\usepackage{caption}
\usepackage{subcaption}
\usepackage[parfill]{parskip}
\usepackage[nottoc]{tocbibind}
\usepackage[backend=biber]{biblatex}
\addbibresource{/home/luispengler/drive/LinuxFabrik/Research/read/bib.bib}
\usepackage{authblk}
\author[1]{Luís Spengler}
\affil[1]{Instituto Federal de Educação, Ciência e Tecnologia de Mato Grosso do Sul}

\begin{document}
\maketitle

\section{Introdução}

\section{Problemática}

\section{Objetivo Geral}

\section{Metodologia}

\section{Resultados}
\begin{tikzpicture}
	\begin{axis}[
		xtick distance = 20,
		ytick distance = 1,
		grid = both,
		minor tick num = 1,
      	        major grid style = {lightgray},
		minor grid style = {lightgray!25},
		width = \textwidth,
		height = 0.5\textwidth,
		xlabel = {$P(mmHg)$},
		ylabel = {$U(V)$},]

		\addplot[
		samples = 200,
		smooth,
		thin,
		red
		] file[skip first] {data.dat};
	\end{axis}
\end{tikzpicture}


\section{Conclusão}

\end{document}
