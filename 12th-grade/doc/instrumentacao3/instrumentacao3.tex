\documentclass{article}
\title{Identificação de histerese em um Esfigmomanômetro}
\date{}

\usepackage[utf8]{inputenc}
\usepackage[portuguese]{babel}
\usepackage[margin=2.5cm]{geometry}
\usepackage{amsmath}
\usepackage{physics}
\usepackage{pgfplots}
\pgfplotsset{compat = newest}
\usepackage{titlesec}
\usepackage{graphicx}
\usepackage{wrapfig}
\usepackage{caption}
\usepackage{subcaption}
\usepackage{tabularx}
\usepackage[parfill]{parskip}
\usepackage[nottoc]{tocbibind}
\usepackage[backend=biber]{biblatex}
\addbibresource{/home/luispengler/drive/LinuxFabrik/Research/read/bib.bib}
\usepackage{authblk}
\renewcommand\Authand{ e }
\renewcommand\Authands{ e }
\author[1]{Luís Spengler}
\author[2]{Gustavo Ratier Cardoso}
\affil[1,2]{Instituto Federal de Educação, Ciência e Tecnologia de Mato Grosso do Sul}

\begin{document}
\maketitle

\section{Introdução}
Após a utilização de um Esfigmomanômetro, embutido em um kit de instrumentação, na sala de Eletrônica, pode-se notar que houve uma perda de pressão do relógio analógico da ferramenta, em termos gerais, denominada Histerese.

A histerese se desdobra pelas mais variadas formas como a magnética, a elétrica e a resistiva, mas trabalharemos com a histerese pneumática, que está relacionada à pressão, normalmente medida em Pascal (P), em função da tensão elétrica, medida em Volts (V).

\section{Problemática}
Identificar e quantificar as perdas (histerese) de pressão (mmHg ou Milímetros de Mercúrio) em função do aumento da tensão elétrica (V).

\section{Objetivo Geral}
Avaliar a possibilidade do instrumento medidor de pressão, o esfigmomanômetro, estar avariado ou fora da capacidade de uso, se tornando inutilizado.

\section{Metodologia}
Com o uso de um multímetro, se mediu a diferença de potencial conforme se pressionava a bomba de látex do esfigmomanômetro, tendo o aumento de tensão elétrica como resposta esperada;

Após o valor visto de 1,42 V mostrado no aparelho multímetro, foi possível notar que a pressão do esfigmomanômetro estava sendo liberada, sem a menor intervenção humana, com uma velocidade relativamente baixa, isso fez com que se fosse abordada a situação de histerese (perda) dos valores do instrumento pneumático.

\section{Resultados}

Por conseguinte, determinamos um gráfico para representarmos a histerese como taxa de perda relacionada à pressão em função da tensão elétrica, após um certo período. Confira os seguintes dados apurados de tensão elétrica (U) e pressão (P), na tabela e no gráfico obtidos abaixo.

\begin{center}
\begin{tabularx}{0.2\textwidth} { 
	| >{\raggedright\arraybackslash}X 
	| >{\centering\arraybackslash}X 
	| >{\raggedleft\arraybackslash}X | }
		\hline
		P & U     \\
		\hline
		20  & 1.42  \\
		40  & 1.83  \\
		60  & 2.4   \\
		80  & 2.9   \\
		100 & 3.48  \\
		120 & 4.07  \\
		140 & 4.71  \\
		160 & 5.31  \\
		180 & 5.98  \\
		200 & 6.61  \\
		220 & 7.23  \\
		240 & 7.9   \\
		260 & 8.57  \\
		280 & 9.3   \\
		300 & 10.03 \\
		\hline
\end{tabularx}
\end{center}
\vspace{5mm}

\begin{tikzpicture}
	\begin{axis}[
		xtick distance = 20,
		ytick distance = 1,
		grid = both,
		minor tick num = 1,
      	        major grid style = {lightgray},
		minor grid style = {lightgray!25},
		width = \textwidth,
		height = 0.5\textwidth,
		xlabel = {$P(mmHg)$},
		ylabel = {$U(V)$},]

		\addplot[
		samples = 200,
		smooth,
		thin,
		red
		] file[skip first] {data.dat};
	\end{axis}
\end{tikzpicture}


\section{Conclusão}
Concluímos que o instrumento necessita sim, de ser verificado e reparado, pós checagens mais aprofundadas, pois a histerese relatada neste caso, influencia de maneira negativa demasiada, afetando o trabalho de medição e gerando desvios e erros desnecessários de medição.

\end{document}
