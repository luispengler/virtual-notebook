\documentclass{article}
\title{Amplificador operacional controlado por variação de resistência de entrada}
\date{}

\usepackage[utf8]{inputenc}
\usepackage[portuguese]{babel}
\usepackage[margin=2.5cm]{geometry}
\usepackage{amsmath}
\usepackage{physics}
\usepackage{pgfplots}
\pgfplotsset{compat = newest}
\usepackage{titlesec}
\usepackage{graphicx}
\usepackage{wrapfig}
\usepackage{caption}
\usepackage{subcaption}
\usepackage{tabularx}
\usepackage[parfill]{parskip}
\usepackage[nottoc]{tocbibind}
\usepackage[backend=biber]{biblatex}
\addbibresource{/home/luispengler/drive/LinuxFabrik/Research/read/bib.bib}
\usepackage{authblk}
\renewcommand\Authand{ e }
\renewcommand\Authands{ e }
\author[1]{Luís Guilherme Miranda Spengler}
\author[2]{Diogo Paes Masacottes}
\affil[1,2]{Instituto Federal de Educação, Ciência e Tecnologia de Mato Grosso do Sul}

\begin{document}
\maketitle

\section{Introdução}

Um amplificador operacional é um amplificador com ganho muito elevado, tendo dois terminais de entrada e um de saída. No circuito performado, na prática, utiliza-se um potenciômetro que varia a tensão do sinal de entrada, para verificar a tensão na saída e observar os efeitos de amplificação. O amplificador operacional, também chamado por alguns de amp-op, nada mais é do que um circuito integrado (CI), capaz de amplificar um sinal de entrada e como próprio nome sugere, o amplificador operacional também consegue realizar operações matemáticas, como, por exemplo, soma, subtração, derivação.

\section{Problemática}

A importância do estudo de amplificadores operacionais se justifica pelo fato de estarmos em constante contato com tal dispositivo eletrônico e também para posterior aplicações como profissionais da área de eletrotécnica.

\section{Objetivo Geral}

Verificar as relações existentes entre a tensão de saída e tensão de entrada conforme a variação do potenciômetro V.S.

\section{Metodologia}

A figura 1 representa o circuito performado, na prática, onde V1 e V2 representam multímetros e suas tensões lidas, AN0 e AN1, respectivamente. Ainda na figura 1, V.S. representará aqui e por todo o artigo, o potenciômetro a ser variado.

\begin{figure}[h!]
\centering
\def\svgwidth{0.6\columnwidth}
\input{ligacao1.pdf_tex}
\caption{Esquema do circuito}
\end{figure}

A regulagem de V.S. foi feita empiricamente, sendo 4 padrões de regulagem definidos:

\begin{itemize}
\item Potenciômetro a 0\%
\item Potenciômetro a 10\%
\item Potenciômetro a 50\%
\item Potenciômetro a 100\%
\end{itemize}

Após a regulagem de V.S. em cada um dos padrões, foi mensurada a tensão em V1 (entrada) e a tensão em V2 (saída).

\section{Resultados}

Por conseguinte, obtivemos dados relacionando a tensão mensurada por V1 e V2 (em Volts) por 4 amostragens, onde a única variável a sofrer alteração foi a resistência na entrada do amplificador (por V.S.).
Confira os seguintes dados apurados da tensão elétrica no resistor de entrada (V.S.) e da tensão elétrica na saída do amplificador operacional, na tabela e no gráfico obtidos abaixo com o potenciômetro a 0\%, 10\%, 50\% e 100\%.

Para 0\%:

\begin{center}
\begin{tabularx}{0.2\textwidth} {
    | >{\raggedright\arraybackslash}X
    | >{\centering\arraybackslash}X
    | >{\raggedleft\arraybackslash}X | }
            \hline
V.S. (V) & SAÍDA (V) \\
\hline
  0.04 & 0.04 \\
  1.03 & 1.04 \\
  2 & 1.78 \\
  3 & 2.92 \\
  4.02 & 3.99 \\
  5.07 & 5.07 \\
  6.03 & 6.06 \\
  7.00 & 7.07 \\
  7.96 & 8.06 \\
  9.06 & 9.18 \\
  10 & 10 \\
\hline
\end{tabularx}
\end{center}
\vspace{1mm}

\begin{tikzpicture}
  \begin{axis}[
        xtick distance = 1,
        ytick distance = 1,
        grid = both,
        minor tick num = 1,
        major grid style = {lightgray},
        minor grid style = {lightgray!25},
        width = \textwidth,
        height = 0.4\textwidth,
        xlabel = {$V.S. (V)$},
        ylabel = {$SAIDA (V)$},]

        \addplot[
        samples = 200,
        smooth,
        thin,
        red
        ] file[skip first] {data0.dat};
      \end{axis}
\end{tikzpicture}

Para 10\%:

\begin{center}
\begin{tabularx}{0.2\textwidth} {
    | >{\raggedright\arraybackslash}X
    | >{\centering\arraybackslash}X
    | >{\raggedleft\arraybackslash}X | }
            \hline
V.S. (V) & SAÍDA (V) \\
  \hline
0 & 0 \\
1.11 & 1.12 \\
1.97 & 1.78 \\
3.03 & 2.98 \\
4.01 & 4.01 \\
4.98 & 5.05 \\
6 & 6.12 \\
6.95 & 7.16 \\
8.05 & 8.27 \\
8.99 & 9.22 \\
10 & 10 \\

\hline
\end{tabularx}
\end{center}
\vspace{5mm}

\begin{tikzpicture}
  \begin{axis}[
        xtick distance = 1,
        ytick distance = 1,
        grid = both,
        minor tick num = 1,
        major grid style = {lightgray},
        minor grid style = {lightgray!25},
        width = \textwidth,
        height = 0.5\textwidth,
        xlabel = {$V.S. (V)$},
        ylabel = {$SAIDA (V)$},]

        \addplot[
        samples = 200,
        smooth,
        thin,
        red
        ] file[skip first] {data10.dat};
      \end{axis}
\end{tikzpicture}

Para 50\%:
\begin{center}
\begin{tabularx}{0.2\textwidth} {
    | >{\raggedright\arraybackslash}X
    | >{\centering\arraybackslash}X
    | >{\raggedleft\arraybackslash}X | }
            \hline
V.S. (V) & SAÍDA (V) \\
  \hline
0.15 & 0.20 \\
1.06 & 1.20 \\
2.02 & 2.17 \\
2.99 & 3.50 \\
4.10 & 4.96 \\
5.06 & 5.17 \\
6.03 & 7.37 \\
8.07 & 9.92 \\
10 & 10 \\
\hline
\end{tabularx}
\end{center}
\vspace{5mm}

\begin{tikzpicture}
  \begin{axis}[
        xtick distance = 1,
        ytick distance = 1,
        grid = both,
        minor tick num = 1,
        major grid style = {lightgray},
        minor grid style = {lightgray!25},
        width = \textwidth,
        height = 0.5\textwidth,
        xlabel = {$V.S. (V)$},
        ylabel = {$SAIDA (V)$},]

        \addplot[
        samples = 200,
        smooth,
        thin,
        red
        ] file[skip first] {data50.dat};
      \end{axis}
\end{tikzpicture}


Para 100\%:

\begin{center}
\begin{tabularx}{0.2\textwidth} {
    | >{\raggedright\arraybackslash}X
    | >{\centering\arraybackslash}X
    | >{\raggedleft\arraybackslash}X | }
            \hline
V.S. (V) & SAÍDA (V) \\
  \hline
1 & 10 \\
2 & 10 \\
3 & 10 \\
4 & 10 \\
5 & 10 \\
6 & 10 \\
7 & 10 \\
8 & 10 \\
9 & 10 \\
10 & 10 \\
\hline
\end{tabularx}
\end{center}
\vspace{5mm}

\begin{tikzpicture}
  \begin{axis}[
        xtick distance = 1,
        ytick distance = 1,
        grid = both,
        minor tick num = 1,
        major grid style = {lightgray},
        minor grid style = {lightgray!25},
        width = \textwidth,
        height = 0.5\textwidth,
        xlabel = {$V.S. (V)$},
        ylabel = {$SAIDA (V)$},]

        \addplot[
        samples = 200,
        smooth,
        thin,
        red
        ] file[skip first] {data100.dat};
      \end{axis}
\end{tikzpicture}

\section{Conclusão}

Na relação dos dados obtidos pelas 4 amostragens entendemos que a resistência é única variável que muda na entrada do amplificador e conforme regulamos o potenciômetro, não percebemos alterações relevantes até a tensão de entrada ser de cinco volts. No entanto, esse comportamento é alterado quando o potenciômetro está em 50\%, e todas as tensões de entrada são menores que as de saídas até chegar a 5V. Contudo, ocorre algo quando o potenciômetro está a 100\% e todas as tensões de saídas são iguais a 10V, independente da tensão de entrada.

\end{document}
