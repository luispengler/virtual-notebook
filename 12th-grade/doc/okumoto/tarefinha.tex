\documentclass{article}
\title{Resolução de Exercícios - Instalações Elétricas Industriais}
\date{}

\usepackage[utf8]{inputenc}
\usepackage[portuguese]{babel}
\usepackage[margin=3.5cm]{geometry}
\usepackage{amsmath}
\usepackage{physics}
\usepackage{titlesec}
\usepackage{graphicx}
\usepackage{wrapfig}
\usepackage{caption}
\usepackage{subcaption}
\usepackage[parfill]{parskip}
\usepackage[nottoc]{tocbibind}
\usepackage[backend=biber]{biblatex}
\addbibresource{/home/luispengler/drive/LinuxFabrik/Research/read/bib.bib}
\usepackage{authblk}
\renewcommand\Authand{ e }
\renewcommand\Authands{ e }
\author[1]{Luís Guilherme Miranda Spengler}
\author[2]{Diogo Paes Masacottes}
\affil[1,2]{Instituto Federal de Educação, Ciência e Tecnologia de Mato Grosso do Sul}

\begin{document}
\maketitle

\section{Exercício 4}
Dados do exercício:
\begin{itemize}

  \item Queda de tensão admissível $=3\%$
  \item Interior de molduras (A1)
  \item Temperatura 35ºC
  \item Isolação dos condutores é EPR
  \item Ft = 0,96 e Fa = 0,8

\end{itemize}

\subsection{Circuito 1}
\begin{itemize}
  \item Critério 1 (Seção mínima): $2,5mm^2$, pois é um circuito de força
  \item Critério 2 (Condução de corrente): $1,5mm^2$, conforme os cálculos e consulta na tabela:

\[I = \frac{S}{V}\]

\[I = \frac{1800}{127} = 14,17 A\]

\[I = \frac{14.17}{Ft\times Fa}\]

\[I = \frac{14.17}{0,96\times 0,8}\]

\[I = \frac{14.17}{0,768} = 18,45A\]

  Seção (de acordo com a tabela 37): $1,5mm^2$
  \item Critério 3 (Queda de tensão): $16mm^2$, conforme os cálculos abaixo:

\[Sc = \frac{200\cdot \rho\cdot l\cdot I_{b}}{\Delta V\cdot V}\]

\[Sc = \frac{200\cdot 0,0178\cdot 120\cdot 14,17}{3\cdot 127} = \frac{6053,424}{381} = 15.88mm^2 \approx 16mm^2\]

\end{itemize}

\textbf{Vale a seção maior, pois atende a todos os critérios: $16mm^2$}

\subsection{Circuito 2}
\begin{itemize}
\item Critério 1 (Seção mínima): $1,5mm^2$, pois é um circuito de iluminação
\item Critério 2 (Condução de corrente): $1,5mm^2$, conforme os cálculos e consulta na tabela:

\[I = \frac{S}{V}\]

\[I = \frac{1650}{127} = 12,99 A\]

\[I = \frac{12,99}{Ft\times Fa}\]

\[I = \frac{12,99}{0,96\times 0,8}\]

\[I = \frac{12,99}{0,768} = 16,91 A\]

Seção (de acordo com a tabela 37): $1,5mm^2$

\item Critério 3 (Queda de tensão): $19mm^2$

\[Sc = \frac{200\cdot \rho\cdot l\cdot I_{b}}{\Delta V\cdot V}\]

        \[Sc = \frac{200\cdot 0,0178\cdot 150\cdot 12,99}{3\cdot 127} = \frac{6936,66}{381} = 18,2mm^2 \approx 19mm^2\]
\end{itemize}

\textbf{Vale a seção maior, pois atende a todos os critérios: $19mm^2$}

\subsection{Circuito 3}

\begin{itemize}
\item Critério 1 (Seção mínima): $2,5mm^2$, pois é um circuito de força
\item Critério 2 (Condução de corrente): $2,5mm^2$, conforme os cálculos e consulta na tabela:

\[I = \frac{S}{V}\]

\[I = \frac{3600}{220} = 16,36 A\]

\[I = \frac{16,36}{Ft\times Fa}\]

\[I = \frac{16,36}{0,96\times 0,8}\]

\[I = \frac{16,36}{0,768} = 21,3 A\]

Seção (de acordo com a tabela 37): $2,5mm^2$

\item Critério 3 (Queda de tensão): $9mm^2$

\[Sc = \frac{200\cdot \rho\cdot l\cdot I_{b}}{\Delta V\cdot V}\]

\[Sc = \frac{200\cdot 0,0178\cdot 100\cdot 16,36}{3\cdot 220} = \frac{5824,16}{660} = 8,82mm^2 \approx 9mm^2\]
\end{itemize}

\textbf{Vale a seção maior, pois atende a todos os critérios: $9mm^2$}

\subsection{Circuito 4}

\begin{itemize}
\item Critério 1 (Seção mínima): $2,5mm^2$, pois é um circuito de força
\item Critério 2 (Condução de corrente): $6mm^2$, conforme os cálculos e consulta na tabela:

\[I = \frac{S}{V}\]

\[I = \frac{6000}{220} = 27,27 A\]

\[I = \frac{27,27}{Ft\times Fa}\]

\[I = \frac{27,27}{0,96\times 0,8}\]

\[I = \frac{27,27}{0,768} = 35,5 A\]

Seção (de acordo com a tabela 37): $6mm^2$

\item Critério 3 (Queda de tensão): $27mm^2$

\[Sc = \frac{200\cdot \rho\cdot l\cdot I_{b}}{\Delta V\cdot V}\]

\[Sc = \frac{200\cdot 0,0178\cdot 180\cdot 27,27}{3\cdot 220} = \frac{17474,616}{660} = 26,47mm^2 \approx 27mm^2\]
\end{itemize}

\textbf{Vale a seção maior, pois atende a todos os critérios: $27mm^2$}

\section{Exercício 5}
Dados do exercício:
\begin{itemize}
\item Queda de tensão admissível $=3\%$
\item No interior de canaletas fechadas embutidas no piso (B1)
\item Temperatura 30ºC
\item Isolação dos condutores é EPR
\item Ft = 0,93 e Fa = 0,7
\end{itemize}

\subsection{Circuito 1}
\begin{itemize}
\item 6000VA, 220V (3 condutores pois é trifásico), L = 200m
\item Critério 1 (Seção mínima): $2,5mm^2$, pois é um circuito de força
\item Critério 2 (Condução de corrente): $6mm^2$, conforme os cálculos e consulta na tabela:

\[I = \frac{S}{V}\]

\[I = \frac{6000}{220} = 27,27 A\]

\[I = \frac{27,27}{Ft\times Fa}\]

\[I = \frac{27,27}{0,93\times 0,7}\]

\[I = \frac{27,27}{0,651} = 41,89A\]

Seção (de acordo com a tabela 37): $6mm^2$

\item Critério 3 (Queda de tensão): $26mm^2$

\[Sc = \frac{173,205\cdot \rho\cdot l\cdot I_{b}}{\Delta V\cdot V}\]

\[Sc = \frac{173,205\cdot 0,0178\cdot 200\cdot 27,27}{3\cdot 220} = \frac{16814,94}{660} = 25,47mm^2 \approx 26mm^2\]
\end{itemize}

\textbf{Vale a seção maior, pois atende a todos os critérios: $26mm^2$}

\subsection{Circuito 2}
\begin{itemize}
\item 9000VA, 220V (3 condutores pois é trifásico), L = 250m
\item Critério 1 (Seção mínima): $2,5mm^2$, pois é um circuito de força
\item Critério 2 (Condução de corrente): $10mm^2$, conforme os cálculos e consulta na tabela:

\[I = \frac{S}{V}\]

\[I = \frac{9000}{220} = 40,9 A\]

\[I = \frac{40,9}{Ft\times Fa}\]

\[I = \frac{40,9}{0,93\times 0,7}\]

\[I = \frac{40,9}{0,651} = 62,84 A\]

Seção (de acordo com a tabela 37): $10mm^2$

\item Critério 3 (Queda de tensão): $48mm^2$

\[Sc = \frac{173,205\cdot \rho\cdot l\cdot I_{b}}{\Delta V\cdot V}\]

\[Sc = \frac{173,205\cdot 0,0178\cdot 250\cdot 40,9}{3\cdot 220} = \frac{31524,17}{660} = 47,76mm^2 \approx 48mm^2\]
\end{itemize}

\textbf{Vale a seção maior, pois atende a todos os critérios: $48mm^2$}

\end{document}
