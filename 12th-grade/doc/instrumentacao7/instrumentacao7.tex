\documentclass{article}
\title{Controle de velocidade de motor baseado em tecnologia pulse-width modulation}
\date{}

\usepackage[utf8]{inputenc}
\usepackage[portuguese]{babel}
\usepackage[margin=2.5cm]{geometry}
\usepackage{amsmath}
\usepackage{physics}
\usepackage{pgfplots}
\pgfplotsset{compat = newest}
\usepackage{titlesec}
\usepackage{graphicx}
\usepackage{wrapfig}
\usepackage{caption}
\usepackage{subcaption}
\usepackage{tabularx}
\usepackage{multirow}
\usepackage[parfill]{parskip}
\usepackage[nottoc]{tocbibind}
\usepackage[backend=biber]{biblatex}
\addbibresource{/home/luispengler/drive/LinuxFabrik/Research/read/bib.bib}
\usepackage{authblk}
\renewcommand\Authand{ e }
\renewcommand\Authands{ e }
\author[1]{Luís Guilherme Miranda Spengler}
\author[2]{Diogo Paes Masacottes}
\affil[1,2]{Instituto Federal de Educação, Ciência e Tecnologia de Mato Grosso do Sul}

\begin{document}
\maketitle

\section{Introdução}
Com a aplicação de um pulso modulado por largura sobre um intervalo de tempo, pode-se reduzir a potência transmitida à carga. Neste relatório descrevemos um experimento de controle de velocidade de um motor baseado na variação de potência transmitida à carga.

\section{Problemática}
A importância do estudo da disciplina de controle se justifica pelo fato de estarmos em constante contato com dispositivos operados por sistemas de controle e também para posterior aplicações como profissionais da área de eletrotécnica.

\section{Objetivo Geral}
Identificar o tipo de controle para incrementar a velocidade do motor, enquanto se observa os efeitos do controle por PWM.

\clearpage
\section{Metodologia}

Utilizando o kit de instrumentação do laboratório, foi possível montar o circuito da figura 1.

\begin{figure}[h!]
\centering
\def\svgwidth{0.6\columnwidth}
\input{ligacao.pdf_tex}
\caption{Esquema do circuito}
\end{figure}

O controle da tensão que vai para o motor DC é feito ajustando o potênciometro de ``Voltage Source'', que fornece só até 10V. O sinal PWM sai de ``PWM Signals''. A velocidade do motor é medida através de um rotary encoder.

Foram medidas as variáveis de tensão, frequência de PWM, Potência (através da relação da tensão e ciclo de trabalho) e frequência de rotação. Os dados obtidos estão na próxima seção.

\section{Resultados}

\begin{center}
\begin{tabular}{ |p{1.5cm}||p{1.7cm}|p{1cm}|p{2.5cm}|  }
 \hline
 \multicolumn{4}{|c|}{Tabela de variáveis} \\
 \hline
 Tensão & Frequência & Pot. & Frequência\\
  \hline
 $0 \approx 10V$ & PWM & \% & Encoder (F/C1)\\
  \hline
  0,02 & 0,10 KHz& 00 & 0 KHz\\
  0,99 & 0,10 KHz& 10 & 0 KHz\\
  1,98 & 0,10 KHz& 20 & 0 KHz\\
  2,99 & 0,10 KHz& 30 & 0 KHz\\
  4,00 & 0,10 KHz& 40 & 0 KHz\\
  5,00 & 0,10 KHz& 50 & 0 KHz\\
  5,92 & 0,12 KHz& 55 & 0,05 KHz\\
  7,01 & 0,10 KHz& 65 & 0,25 KHz\\
  7,97 & 0,10 KHz& 75 & 0,55 KHz\\
  8,92 & 0,10 KHz& 85 & 0,95 KHz\\
  10,0 & 0,10 KHz& 100 & 1,65 KHz\\
 \hline
\end{tabular}
\end{center}

\section{Conclusão}
Como é necessário incrementar a tensão na entrada do motor para ter uma maior velocidade (neste caso medida em termos de frequência), pode-se concluir que estamos trabalhando com um sistema em malha fechada manual, enquanto se aprofunda os conhecimentos técnicos de encoders e PWM da disciplina.

\end{document}
