%% abtex2-modelo-artigo.tex, v-1.9.7 laurocesar
%% Copyright 2012-2018 by abnTeX2 group at http://www.abntex.net.br/ 
%%
%% This work may be distributed and/or modified under the
%% conditions of the LaTeX Project Public License, either version 1.3
%% of this license or (at your option) any later version.
%% The latest version of this license is in
%%   http://www.latex-project.org/lppl.txt
%% and version 1.3 or later is part of all distributions of LaTeX
%% version 2005/12/01 or later.
%%
%% This work has the LPPL maintenance status `maintained'.
%% 
%% The Current Maintainer of this work is the abnTeX2 team, led
%% by Lauro César Araujo. Further information are available on 
%% http://www.abntex.net.br/
%%
%% This work consists of the files abntex2-modelo-artigo.tex and
%% abntex2-modelo-references.bib
%%

% ------------------------------------------------------------------------
% ------------------------------------------------------------------------
% abnTeX2: Modelo de Artigo Acadêmico em conformidade com
% ABNT NBR 6022:2018: Informação e documentação - Artigo em publicação 
% periódica científica - Apresentação
% ------------------------------------------------------------------------
% ------------------------------------------------------------------------

\documentclass[
	% -- opções da classe memoir --
	article,			% indica que é um artigo acadêmico
	11pt,				% tamanho da fonte
	oneside,			% para impressão apenas no recto. Oposto a twoside
	a4paper,			% tamanho do papel. 
	%twocolumn,
	% -- opções da classe abntex2 --
	%chapter=TITLE,		% títulos de capítulos convertidos em letras maiúsculas
	%section=TITLE,		% títulos de seções convertidos em letras maiúsculas
	%subsection=TITLE,	% títulos de subseções convertidos em letras maiúsculas
	%subsubsection=TITLE % títulos de subsubseções convertidos em letras maiúsculas
	% -- opções do pacote babel --
	english,			% idioma adicional para hifenização
	brazil,				% o último idioma é o principal do documento
	sumario=tradicional
	]{abntex2}


% ---
% PACOTES
% ---

% ---
% Pacotes fundamentais 
% ---
\usepackage{lmodern}			% Usa a fonte Latin Modern
\usepackage[T1]{fontenc}		% Selecao de codigos de fonte.
\usepackage[utf8]{inputenc}		% Codificacao do documento (conversão automática dos acentos)
\usepackage{indentfirst}		% Indenta o primeiro parágrafo de cada seção.
\usepackage{nomencl} 			% Lista de simbolos
\usepackage{color}				% Controle das cores
\usepackage{graphicx}			% Inclusão de gráficos
\usepackage{microtype} 			% para melhorias de justificação
% ---
		
% ---
% Pacotes adicionais, usados apenas no âmbito do Modelo Canônico do abnteX2
% ---
\usepackage{lipsum}				% para geração de dummy text
% ---
		
% ---
% Pacotes de citações
% ---
\usepackage[brazilian,hyperpageref]{backref}	 % Paginas com as citações na bibl
\usepackage[alf]{abntex2cite}	% Citações padrão ABNT
% ---

% ---
% Configurações do pacote backref
% Usado sem a opção hyperpageref de backref
\renewcommand{\backrefpagesname}{Citado na(s) página(s):~}
% Texto padrão antes do número das páginas
\renewcommand{\backref}{}
% Define os textos da citação
\renewcommand*{\backrefalt}[4]{
	\ifcase #1 %
		Nenhuma citação no texto.%
	\or
		Citado na página #2.%
	\else
		Citado #1 vezes nas páginas #2.%
	\fi}%
% ---

% --- Informações de dados para CAPA e FOLHA DE ROSTO ---
\titulo{Fundamentos teóricos: regras e técnicas de badminton}
\tituloestrangeiro{Theoretical foundation: Badminton's rules and technics}

\autor{Luís Guilherme Miranda Spengler}

\local{Campo Grande, MS, Brasil}
\data{2022}
% ---

% ---
% Configurações de aparência do PDF final

% alterando o aspecto da cor azul
\definecolor{blue}{RGB}{41,5,195}

% informações do PDF
\makeatletter
\hypersetup{
     	%pagebackref=true,
		pdftitle={\@title}, 
		pdfauthor={\@author},
    	pdfsubject={Modelo de artigo científico com abnTeX2},
	    pdfcreator={LaTeX with abnTeX2},
		pdfkeywords={abnt}{latex}{abntex}{abntex2}{atigo científico}, 
		colorlinks=true,       		% false: boxed links; true: colored links
    	linkcolor=blue,          	% color of internal links
    	citecolor=blue,        		% color of links to bibliography
    	filecolor=magenta,      		% color of file links
		urlcolor=blue,
		bookmarksdepth=4
}
\makeatother
% --- 

% ---
% compila o indice
% ---
\makeindex
% ---

% ---
% Altera as margens padrões
% ---
\setlrmarginsandblock{3cm}{3cm}{*}
\setulmarginsandblock{3cm}{3cm}{*}
\checkandfixthelayout
% ---

% --- 
% Espaçamentos entre linhas e parágrafos 
% --- 

% O tamanho do parágrafo é dado por:
\setlength{\parindent}{1.3cm}

% Controle do espaçamento entre um parágrafo e outro:
\setlength{\parskip}{0.2cm}  % tente também \onelineskip

% Espaçamento simples
\SingleSpacing


% ----
% Início do documento
% ----
\begin{document}

% Seleciona o idioma do documento (conforme pacotes do babel)
%\selectlanguage{english}
\selectlanguage{brazil}

% Retira espaço extra obsoleto entre as frases.
\frenchspacing 

% ----------------------------------------------------------
% ELEMENTOS PRÉ-TEXTUAIS
% ----------------------------------------------------------

%---
%
% Se desejar escrever o artigo em duas colunas, descomente a linha abaixo
% e a linha com o texto ``FIM DE ARTIGO EM DUAS COLUNAS''.
% \twocolumn[    		% INICIO DE ARTIGO EM DUAS COLUNAS
%
%---

%\twocolumn[              % INICIO DE ARTIGO EM DUAS COLUNAS

% página de titulo principal (obrigatório)
\maketitle


% titulo em outro idioma (opcional)


% resumo em português
\begin{resumoumacoluna}
 Este presente trabalho foca em descrever os fundamentos teóricos e avança o progresso 
do estudante interessado em aprender sobre as regras, técnicas e golpes do Badminton.
Esse esporte não possui muitas regras complicadas de se entender e técnicas fora do
normal, quando um observa fundamentos de outros esportes similares (voleibol, tênis de quadra, etc).
Também vai-se perdendo uma cultura eletista no sentido de possuir equipamentos especiais,
espaços grandes e abertos, ou grandes e cobertos; o esporte se torna mais inclusivo e 
auxilia na saúde física de todas as faixas etárias. Com esses pontos, se justifica a
produção de manuais úteis que auxiliem no ensinamento do fundamento teórico do badminton.

 \vspace{\onelineskip}
 
 \noindent
 \textbf{Palavras-chave}: badminton. regras do badminton. técnicas do badminton.
\end{resumoumacoluna}


% resumo em inglês
\renewcommand{\resumoname}{Abstract}
\begin{resumoumacoluna}
 \begin{otherlanguage*}{english}
This following paper focus on describing the theoretical foundation and improves the progress
of the student interested in learning about the badminton's rules, technics, and moves.
This sport does not possue many complicated rules and technics out of the scope, when one
observes other similar sports (volleyball, tennis, etc). There is also an eletist culture
that loses itself in the sense of owning special equipment, big and open spaces, or big and covered ones;
the sport becomes more inclusive and it has direct effects on the physical health of all age groups.
With these points being made, it justifies the production of useful manuals that help the badminton's theoretical foundation
teaching.

   \vspace{\onelineskip}
 
   \noindent
   \textbf{Keywords}: badminton. badminton's rules. badminton's technics.
 \end{otherlanguage*}  
\end{resumoumacoluna}

  				% FIM DE ARTIGO EM DUAS COLUNAS
---

\begin{center}\smaller
\textbf{Data de submissão e aprovação}: Data de submissão: 18 de março de 2022. Data de aprovação: Pendente.

\textbf{Identificação e disponibilidade}: luis.spengler@estudante.ifms.edu.br 
\end{center}

% ----------------------------------------------------------
% ELEMENTOS TEXTUAIS
% ----------------------------------------------------------
\textual

% ----------------------------------------------------------
% Introdução
% ----------------------------------------------------------
\section{Introdução}
Badminton trata-se de um esporte, que, embora leve em consideração resistência e agilidade no que diz respeito aos atributos de um bom jogador, permite uma prática de diversos grupos de idade e habilidades físicas.

O mesmo não possui fundamentos difíceis de dominar, nem muitos equipamentos para principiantes no esporte. Com uma disposição a fazer-se entender das regras, pode-se disfrutar dos benefícios da prática do Badminton em um periodo de tempo mais curto em comparação com outros esportes jogados de forma similar.

% ----------------------------------------------------------
% Seção de explicações
% ----------------------------------------------------------

\setlrmarginsandblock{3cm}{3cm}{*}
\setulmarginsandblock{3cm}{3cm}{*}
\checkandfixthelayout

\section{Regras}
\subsection{No que diz respeito ao começo do jogo}
\begin{enumerate}
	\item Um sorteio com uma moeda ou com a peteca de jogo é realizada. O vencedor opta por servir (chamado de saque), receber ou escolhe um dos lados da quadra.
	\item O atleta que serve deve permanecer dentro da área de serviço no lado direito da quadra e fitando a rede. O atleta recebedor fica do outro lado da rede permanecendo dentro da área de serviço no lado direito da quadra, na diagonal de quem serve. Nos jogos em duplas, o parceiro pode ficar em qualquer lugar da quadra desde que não bloqueie a visão do recebedor. 
	\item Previamente a partida, os atletas têm direito a um aquecimento de dois minutos antes do jogo.
\end{enumerate}

\subsection{No que diz respeito à posição durante o jogo}
\begin{enumerate}
	\item Se o placar de quem serve for par, o serviço deve ser feito do lado direito da quadra. Se o placar for ímpar, do lado esquerdo da quadra. Nos jogos em duplas a regra é a mesma. 
	\item O servidor permanece servindo sempre que ele ou sua dupla ganhar o rally. 
\end{enumerate}

\subsection{No que diz respeito ao tempo durante o jogo}
\begin{enumerate}
	\item Sempre que o 1º jogador/dupla atingir 11 pontos um tempo de 60 segundo é concedido. Esta regra vale para qualquer game.
	\item Nos intervalos do 1º para o 2º game e do 2º para o 3º game (se houver) um intervalo de dois minutos é concedido. 
\end{enumerate}

\subsection{No que diz respeito ao serviço}
\begin{enumerate}
	\item Saques no Badminton são sempre realizados na diagonal. 
	\item O serviço, tanto no jogo de simples quanto no de duplas, inicia-se pelo lado direito da quadra de quem serve que deve lançar a peteca obliquamente sobre a rede, para o seu lado esquerdo da quadra adversária.
	\item O recebedor não deve se mexer até que quem serve golpeie a peteca
	\item Ao vencer o ponto, o mesmo jogador continua servindo, mas sua posição deve ser invertida na quadra.
	\item Havendo perda do ponto, o serviço passa para o lado adversário.
	\item Se um erro de área de serviço for cometido, o erro não será corrigido e o jogo continuará sem mudança na área de serviço dos jogadores. 
\end{enumerate}

\subsection{No que diz respeito às faltas}
\begin{enumerate}
	\item Se o atleta (raquete ou roupa inclusive) encostar na rede enquanto a peteca está em jogo será considerado falta.
	\item Se a peteca acerta o jogador, sua roupa, teto ou arredores da quadra será considerado falta.
	\item Se a peteca cair fora das linhas da quadra (a linha é considerada parte da quadra) será considerado falta.
	\item Se o jogador invade ou acerta a peteca no lado oposto da rede (não vale 'carregar' a peteca) será considerado falta.
	\item Se a peteca for golpeada duas vezes do mesmo lado da quadra será considerado falta.
	\item Se houver interferência com a peteca, mau comportamento ou 'cera', o jogador perde o serviço ou o oponente ganha um ponto
	\item Se o parceiro do recebedor receber o serviço será considerado falta.
	\item Se o servidor faz o movimento e erra a peteca será considerado falta.
\end{enumerate}

\subsection{No que diz respeito ao fim do jogo}
\begin{enumerate}
	\item Os jogos são disputados num total três games. O vencedor é o que ganhar dois games primeiro. Em todas as modalidades, os games são de 21 pontos. 
	\item Se houver empate em 20 pontos, vencerá aquele que abrir 2 pontos de vantagem. Havendo empate em 29, vencerá aquele que fizer 30 pontos. 
	\item O jogador que venceu o primeiro game serve primeiro do outro lado da quadra no novo game. O ganhador do segundo game muda de lado e começa servindo no terceiro game. No terceiro game, o jogador muda de lado e continua servindo no décimo primeiro ponto. 
\end{enumerate}

\section{Técnicas e principais golpes}
\subsection{Saque/Serviço}
\subsubsection{Deve se}
\begin{enumerate}
	\item manter os dois pés ou parte deles no chão em uma posição imóvel;
	\item acertar primeiramente a base do volante (peteca);
	\item golpear o volante (peteca) abaixo da sua linha de cintura;
	\item tocar o volante (peteca) abaixo da linha da mão que segura a raquete;
	\item manter o movimento contínuo da raquete, nunca enganar o oponente.
\end{enumerate}

\subsection{Empunhadura}
\begin{enumerate}
	\item A raquete deve ser segurada como se tivesse apertando a mão de alguém.
	\item Os movimentos de punho participam bastante do badminton.
	\item Forehand se refere a golpes realizados com a palma da mão virada para frente (ou para a rede).
	\item Backhand se refere a golpes realizados com as ``costas'' da mão virada para frente (ou para rede).
\end{enumerate}

\subsection{Clear}
\begin{enumerate}
	\item É um dos fundamentos básicos do badminton e consiste em o jogador bater na peteca com a raquete em direção ao fundo da quadra de seus adversários, quando a peteca estiver acima e à frente da cabeça deste jogador.
\end{enumerate}

\subsection{Drop/Drop-Shot}
\begin{enumerate}
	\item É um golpe de ataque no badminton, cujo objetivo é lançar a peteca próximo à rede, dificultando a defesa do seu adversário.
\end{enumerate}

\subsection{Smash}
\begin{enumerate}
	\item É o golpe de ataque mais veloz e mais importante do badminton.
	\item Sua execução é feita golpeando a peteca com precisão e força, de cima para baixo.
	\item Pode ser feito com ou sem salto, a depender da altura da peteca no momento de realização do golpe.
\end{enumerate}

\subsection{Drive}
\begin{enumerate}
	\item Golpe realizado quando a peteca está a meia altura, o que seria mais ou menos na altura dos ombros do jogador.
	\item É executado como se o jogador estivesse chicoteando a peteca.
	\item O objetivo desse golpe é lançar a peteca em um ponto "vazio" da quadra adversária.
\end{enumerate}

\subsection{Lob}
\begin{enumerate}
	\item É um golpe utilizado quando a peteca é lançada bem próximo a rede. Para se defender o jogador de badminton bate na peteca de baixo pra cima, para ganhar altura e cair no fundo da quadra adversária.
	\item Um lob característico tem altura boa, boa profundidade e boa recomposição em quadra após o golpe.
\end{enumerate}

\subsection{Net-Shot/Net-Drop}
\begin{enumerate}
	\item Assim como Lob, este é utilizado quando a peteca é lançada próxima a rede, só que com a diferença que no Net-Shot a peteca é lançada perto da rede do lado do adversário.
	\item Este golpe pode ser utilizado tanto de baixo para cima, quanto de cima para baixo.
	\item Ao realizar o Net-Shot, é importante a recuperação rápida.
\end{enumerate}

% ---
% Finaliza a parte no bookmark do PDF, para que se inicie o bookmark na raiz
% ---
\bookmarksetup{startatroot}% 
% ---

% ---
% Conclusão
% ---
\section{Considerações finais}
Por experiência própria, digo que é possível entender boa parte das regras que compõe o esporte em poucas horas. O mesmo foi feito por vários estudantes da minha turma durante as aulas de Educação Física do professor Sinésio. E com a fundamentação teórica, que consiste na pesquisa feita posteriormente, é possível esclarecer pontos que as aulas práticas não cobriram e esta deve ser consultada, havendo necessidade. 

%\begin{citacao} % Para citações
%\lipsum[2]
%\end{citacao}

% ----------------------------------------------------------
% ELEMENTOS PÓS-TEXTUAIS
% ----------------------------------------------------------
% ----------------------------------------------------------
% Referências bibliográficas
% ----------------------------------------------------------

\end{document}
