\documentclass{article}
\title{Sistema de controle de mesa de som - Sistemas digitais e microcontrolados}
\date{}

\usepackage[utf8]{inputenc}
\usepackage[portuguese]{babel}
\usepackage[margin=3.5cm]{geometry}
\usepackage{amsmath}
\usepackage{physics}
\usepackage{titlesec}
\usepackage{graphicx}
\usepackage{wrapfig}
\usepackage{caption}
\usepackage{subcaption}
\usepackage{karnaugh-map}
\usepackage[parfill]{parskip}
\usepackage[nottoc]{tocbibind}
\usepackage[backend=biber]{biblatex}
\addbibresource{/home/luispengler/drive/LinuxFabrik/Research/read/bib.bib}
\usepackage{authblk}
\author[1]{Giovanna Bughi}
\author[2]{Gustavo Ratier Cardoso}
\author[3]{João Vitor Medeiros}
\author[4]{Luís Spengler}
\affil[1,2,3,4]{Instituto Federal de Educação, Ciência e Tecnologia de Mato Grosso do Sul}

\graphicspath{{./docs/}}

\begin{document}
\maketitle

\tableofcontents

\medskip

\section{Problema proposto}
Uma mesa de som conecta três microfones numa única caixa de som amplificada, que são: Microfone Presidente, Microfone Diretor e Microfone Coordenador. Sabendo que somente um microfone pode falar por vez. Elabore um circuito lógico combinacional que permita ligar os microfones segundo a prioridade abaixo:

Prioridade 1 : Presidente         

Prioridade 2 : Diretor

Prioridade 3 : Coordenador

Cada Microfone é acionado pelo usuário por um interruptor  ( liga-desliga) (ChP, ChD, ChC). Cada microfone  ao ser acionado tem sua saída  comutada (0 ou 1) informando ao circuito lógico,  que por sua vez, aciona uma das  saídas (SP, SD, SC), para a caixa amplificada. Então, quando o Presidente ligar seu microfone, terá prioridade sobre os demais. Quando o Diretor ligar seu microfone só terá prioridade sobre o  Coordenador. O Coordenador só fala quando os demais não estiverem com seus microfones ligados.

\subsection{Esboço do esquema proposto}
O problema pode ser esboçado de acordo com o texto acima.

\includegraphics[width=\textwidth]{esbocoroubado}

\subsection{Definição das variáveis de entrada e saída e seus estados em uma tabela verdade}
Foi considerado cada usuário do microfone como uma variável, estes sendo as variáveis de entrada (ChP, ChD, ChC). As suas respectivas saídas (SP, SD, SC) foram definidas como variáveis de saída. Obedecendo a prioridade de cada falante, as saídas na tabela verdade abaixo, podem ter seus estados definidos.
\begin{displaymath}
\begin{array}{|c c c|c c c|}
INPUT & & & OUTPUT &\\
\hline
ChP & ChD & ChC & SP & SD & SC\\
\hline % Put a horizontal line between the table header and the rest.
0 & 0 & 0 & 0 & 0 & 0\\
0 & 0 & 1 & 0 & 0 & 1\\
0 & 1 & 0 & 0 & 1 & 0\\
0 & 1 & 1 & 0 & 1 & 0\\
1 & 0 & 0 & 1 & 0 & 0\\
1 & 0 & 1 & 1 & 0 & 0\\
1 & 1 & 0 & 1 & 0 & 0\\
1 & 1 & 1 & 1 & 0 & 0\\
\end{array}
\end{displaymath}

\subsection{Obtenção da expressão de saída}

\subsection{Mapa de Karnaugh}
Mapa de Karnaugh para a saída do presidente (SP)

\begin{karnaugh-map}*[4][2][1][$ChPChD$][$ChC$]
	\minterms{2,3,6,7}
	\maxterms{0,1,4,5}
	\indeterminants{2,5}
	\implicant{3}{2}
	\implicant{7}{6}
\end{karnaugh-map}

Mapa de Karnaugh para a saída do diretor (SD)

\begin{karnaugh-map}*[4][2][1][$ChPChD$][$ChC$]
	\maxterms{0,2,3,4,6,7}
	\minterms{1,5}
	\implicant{1}{5}
\end{karnaugh-map}

Mapa de Karnaugh para a saída do coordenador (SC)

\begin{karnaugh-map}*[4][2][1][$ChPChD$][$ChC$]
	\maxterms{0,1,2,3,5,6,7}
	\minterms{4}
	\implicant{4}{4}
\end{karnaugh-map}

\subsection{Simplificação da expressão pelo mapa de Karnaugh}

\subsection{Obtenção do circuito lógico}

\subsection{Implementação do hardware a partir do circuito lógico}

\section{Conclusão}

\medskip

\end{document}
